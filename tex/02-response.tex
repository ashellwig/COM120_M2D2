%! TEX root=../main.tex

\subsection{Response 2}
  \begin{quotation}
    Of the many skills of interpersonal communication that I want to improve is
      Assertiveness. According to the book I read Assertiveness includes when
      you confidently express your needs and opinions in a fair, honest and
      calm way whilst considering the needs and views of other people. As far
      as I know about myself I am not good at expressing what I want, I back
      off more and give in than telling my teammates what I want, sometimes I
      think if I tell what I want it is more likely to be rejected, rejection
      is things that I still fear today. That's why I still want to learn to
      express myself.

    Besides the Assertiveness that I want to achieve in this interpersonal
      communication class, I also want to improve my ability in one more skill,
      namely empathy. I feel that when I work with someone I often don't care
      about how my coworkers feel because sometimes I'm too logical. To be
      empathetic means that you are able to identify and understand others'
      emotions, often this is not what I do. I often feel that feelings are not
      necessary in making decisions. From now on I want to improve my ability to
      understand my colleagues and be more empathetic.
  \end{quotation}

  \paragraph{This is a response to Marcella Ferchette on Post ID 43336346}
    I enjoyed reading your post again, Marcella! In regards to your attempt to
      become more empathetic and ``less logical'', I wanted to offer one piece
      of advice maybe you could use in your practicing.

    While empathy exclusively deals with the emotional intelligence of an
      individual, in reality, being empathetic and providing others with
      sympathy \textit{is} the logical thing to do. Treating people as a
      \textit{means to an end}, rather than as the end itself, is inherently
      bad practice. This is especially important in the work place, as often
      the relationships you build will dictate your future in that industry
      more than experience and school ever will. People would much rather
      give their friend and confidant a raise or promotion, rather than someone
      they rarely speak to!
