%! TEX root=../main.tex

\section{Responses}
  \subsection{Response 1}
    \begin{quotation}
      At the beginning of this course I chose to improve my ability as an
        engaged listener and asserting myself. These were two abilities that due
        to some interesting events this past weekend, I was able to practice for
        many hours than I anticipated to.

      I watched that clip from The Break-up and I want to say they were both
        trying to gain power over each other, the ``win-win approach''. You can
        tell as both Vince Vaughn's character and Jennifer Aniston's character
        saw each other as opponents, like it was a game to see who was more
        right than the other person, and it was as if the actual break-up that
        occurs was like them admitting that the game was no longer fun to play,
        there was no point in trying to win it anymore.

      I can't say that conflict is a strong thing on my list but I think
        referring to the content in the CR List that lists each conflict
        resolution method and helps you identify the ones you use, or have used
        in the past with certain individuals, helps you put things into
        perspective of how you're dealing with these conflicts and what other
        methods you should try practicing with.

      I can't stress enough that all this communication stuff really comes down
        to confidence. The link I'll attach at the end will give you twenty-five
        tips on how to build your confidence. Personally, I think some of these
        ``build your confidence'' websites are a little ridiculous, but in a
        weird way I think they're meant to be ridiculous because if you're
        confidence enough, you can be as ridiculous as you want and still feel
        good about yourself. Kind of like if you think wearing purple pants is
        ridiculous, you'll eventually build the confidence someday to wear those
        purple pants without even worrying about what other people think, or
        you're confident enough that other people thinks it's cool now to wear
        purple pants.

      It really does come down to feeling confident enough to where if conflict
        is on your list then you'll feel confident with the resolution methods
        you've practice, and the confidence you've gained, than any conflict can
        be easily navigable.
    \end{quotation}

    \paragraph{This is a response to Lillian Morton on Post ID 43272328}
      Great read, Lillian! I wholeheartedly agree with the sentiment that
        confidence goes a \textit{very long way} in terms of navigating a
        difficult conflict. The phrase \textit{``fake it till you make it''}
        comes to mind!

      I dealt with a similar situation to your ``purple pants'' situational
        theory! I was born male and I identify as \textit{gender fluid}. Raves
        and music festivals tend to be the only place I feel safe enough to
        express my feminine side, though. It took a long time to build the
        confidence to dress femininely in public when I feel more feminine, but
        with the support of some close friends I was able to attend my first
        show --- GRiZ at Red Rocks --- in full hair and makeup and dance with my
        friends, who dedicated their time to having fun with the group as well
        as preventing any unforeseen conflicts from occurring from any bigoted
        individuals around. It almost seems like if you \textit{act} like
        you are supposed to be there and be there that way, other people will
        simply assume you are!
